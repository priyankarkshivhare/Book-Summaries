\section*{Antiaging Secrets: Little Things That Add Up to a Long and Happy Life}

Neglecting mental exercise can lead to the \textbf{deterioration of neurons and neural connections}. Engaging in new experiences, continuous learning, and social interactions fosters cognitive health and slows mental aging. 

\textbf{Stress} is a major factor that \textbf{accelerates aging}, however, it can be managed through practicing \textbf{mindfulness}. While a \textbf{little stress} can be healthy and motivating, \textbf{chronic stress} takes a serious toll on both the body and mind.

\textbf{Physical movement} is emphasized—\textbf{excessive sitting} is linked to premature aging, so incorporating \textbf{light activity throughout the day} is essential. 

\textbf{Quality sleep} allows the body to \textbf{regenerate}, supports \textbf{immune function}, and boosts \textbf{melatonin production}, a hormone known to \textbf{slow the aging process}. 

A \textbf{positive, calm attitude} — rooted in \textbf{emotional resilience} and \textbf{stoic thinking}(\textit{calmness during adversity}) — can help reduce anxiety and contribute to a \textbf{longer, more fulfilling life}.

