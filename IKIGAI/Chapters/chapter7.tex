\section*{The Ikigai Diet: What the World's Longest-Living People Eat and Drink}
\textbf{\textit{Core Principles of the Ikigai Diet}}:
\begin{itemize}
    \item \textbf{\textit{Plant-Based and Nutrient-Rich Foods}}: The Okinawan diet emphasizes a diverse intake of vegetables, legumes, and fruits. Staples include sweet potatoes, bitter melon (goya), seaweed (kombu, nori), tofu, miso, and soy sprouts. These foods are rich in antioxidants and essential nutrients that support overall health.
    \item \textbf{\textit{Whole Grains as Dietary Staples}}: Grains, particularly white rice, form the foundation of meals. Whole grains like brown rice and barley are also consumed, providing sustained energy and fiber.
    \item \textbf{\textit{Moderate Protein Intake}}: Protein sources primarily come from fish, consumed about three times a week, and soy-based products. Red meat and processed foods are eaten sparingly, reducing the risk of chronic diseases.
    \item \textbf{\textit{Minimal Sugar and Salt Consumption}}: Sugar is rarely used, and when it is, cane sugar is preferred. Salt intake is kept below 10 grams per day, aligning with health guidelines to prevent hypertension.
    \item \textbf{\textit{Herbal Teas and Hydration}}: Green tea, White Tea and other herbal teas are integral to the diet, offering antioxidants and promoting hydration. These beverages are linked to various health benefits, including improved circulation and bone health.
\end{itemize}
