\section*{\textit{IKIGAI}: The art of staying young while growing old}

\textbf{\textit{IKIGAI}}: In contrast to many Western societies, where \textbf{retirement often signals a withdrawal from work and purpose}, the Japanese stay active and purposeful well into their 90s. The concept of “retirement” doesn't exist in Japanese culture. In fact, \textbf{there is no exact word in Japanese for “retire”}. 

\textbf{\textit{Hara Hachi Bu}}: \textbf{“Eat until you are 80\% full.”} It’s a mindful approach to eating that helps prevent overeating, reduces stress on the digestive system, and slows down cellular oxidation and aging. 
\begin{itemize}
    \item Meals are traditionally served in at least five small dishes, \textbf{encouraging variety and smaller portions.}
\end{itemize}

\textbf{\textit{Moai}}: A traditional Okinawan concept referring to informal, lifelong social support groups formed around shared interests or neighborhoods. Members of a moai regularly meet to offer each other emotional, social, and even financial support when needed. This deep-rooted culture of community fosters a strong sense of \textbf{belonging}, emotional well-being, and long-term \textbf{security}, all of which are linked to greater longevity and happiness.
