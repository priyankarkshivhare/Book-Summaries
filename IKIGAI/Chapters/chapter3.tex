\section*{From Logotherapy to Ikigai – How to Live Longer and Better by Finding Your Purpose}

The authors delve into two psychotherapies—\textbf{Logotherapy} and \textbf{Morita therapy}—both grounded in the personal quest for meaning and purpose. Unlike psychoanalysis, which delves into past experiences and subconscious thoughts, they focus on action and acceptance rather than past trauma, guiding individuals toward their \textbf{ikigai}.

\textit{\textbf{Logotherapy}}: Developed by Viktor Frankl — a Holocaust survivor and psychiatrist — \textbf{logotherapy} centers on the belief that humanity’s primary drive is the \textbf{search for meaning}. It emphasizes the present and future, helping individuals uncover purpose even in the harshest conditions. Frankl introduced \textbf{existential frustration}—the distress caused by a lack of meaning—not as a pathology but as a catalyst for growth and self-discovery.

\textit{\textbf{Morita Therapy}}: Developed by Japanese psychiatrist Shoma Morita, \textbf{Morita therapy} emphasizes accepting one's emotions without attempting to control or suppress them. It teaches that while we cannot always control our feelings, we can control our actions. By accepting emotions as natural responses and focusing on purposeful behavior, individuals can lead more authentic and fulfilling lives.

\textit{\textbf{Naikan Meditation}} developed by Yoshimoto Ishin, is a self-reflection practice based on three introspective questions:
\begin{enumerate}
    \item What have I received from person X?
    \item What have I given to person X?
    \item What troubles have I caused person X?
\end{enumerate}
It cultivates gratitude, empathy, and awareness by encouraging individuals to recognize the support they've received and reflect on their impact on others.