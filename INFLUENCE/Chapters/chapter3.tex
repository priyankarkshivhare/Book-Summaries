\section{Commitment and Consistency - Hobgoblins of the Mind}

\subsection*{Overview}
The psychological principle of \textbf{commitment and consistency} highlights the powerful drive for individuals to align their behaviors with previously made commitments. Rooted in the need to appear rational and dependable, this principle compels people to remain consistent with their past actions and statements, even when doing so may lead to irrational decisions.

Understanding this principle enables individuals and organizations to leverage it for constructive purposes—such as reinforcing positive habits—or to recognize and resist manipulative tactics that exploit it.


\subsection*{Key Concepts and Principles}
\begin{enumerate}
    \item \textbf{Commitment} is the foundation of this principle. Once a person commits to something, they are more likely to follow through with it, particularly if the commitment is public or requires effort.
    
    \textbf{Examples}:
    \begin{itemize}
        \item \textbf{Horse Betting:} When individuals place a bet on a horse, their confidence in their choice often increases, illustrating how commitment alters perception and decision-making.
        \item \textbf{Toy Advertising:} Toy companies exploit commitment by advertising limited-stock toys before the holidays, knowing that parents who have committed to buying these toys will be more likely to purchase them at a higher price later.
    \end{itemize}
    \item The \textbf{foot-in-the-door} technique is a persuasion strategy where a small initial request is made to secure agreement, with the intention of following up with a larger request. Once a person agrees to a small commitment, they are more likely to agree to subsequent, larger requests in order to remain consistent with their previous actions.

    \textbf{Examples}:
    \begin{itemize}
        \item A salesperson may first ask for a small favor or purchase, and after gaining agreement, they request a larger purchase.
        \item Chinese POW Camps: During the Korean War, Chinese captors used the principle of commitment and consistency to manipulate American prisoners of war by first obtaining small compliance requests, which led to greater ideological shifts over time.
    \end{itemize}
    \item \textbf{Public commitments} increase the likelihood of compliance with future requests. When individuals commit to something publicly, the pressure to remain consistent with that commitment is amplified by social expectations.

    \textbf{Examples}:
    \begin{itemize}
        \item Announcing a goal in front of others or on social media increases the commitment to achieving that goal due to the public nature of the promise.
    \end{itemize}
    \item The \textbf{effort justification} effect occurs when people perceive greater value in a commitment that required substantial effort to achieve. This principle suggests that the more effort one invests in a particular course of action, the more valuable that action will appear, often regardless of the actual outcome.

    \textbf{Examples}:
    \begin{itemize}
        \item People who undergo arduous initiation rituals to join a group may place greater value on the group and its associated beliefs due to the effort they invested.
    \end{itemize}
    \item The \textbf{low-balling technique} involves securing an initial agreement to a favorable deal, only to later increase the cost or alter the terms. Individuals who have committed to the initial agreement often proceed with the revised, less favorable deal due to their need for consistency.

    \textbf{Examples}:
    \begin{itemize}
        \item A car dealership may offer a customer an attractive price for a car, and once the customer commits, the price may be raised or additional fees added, but the customer proceeds with the purchase because of their initial commitment.
    \end{itemize}
    \item The \textbf{act of writing} down a commitment enhances \textbf{personal responsibility} and reinforces the intention to follow through, as written commitments create a record that can be reviewed later, increasing the pressure to act consistently with the written word.\\
    \textbf{Examples}:
    \begin{itemize}
        \item People who write down their goals or sign contracts are more likely to follow through with their actions than those who make verbal commitments alone.
    \end{itemize}
\end{enumerate}


\subsection*{Defense Against Manipulation}
Strategies to resist undue influence from the commitment principle:
\begin{itemize}
    \item \textbf{Internal Warning Signs:} A sense of discomfort or unease when making a decision may indicate that the commitment is being used manipulatively. Individuals should pay attention to these ``stomach signs''.
    \item \textbf{Reflection on Commitment:} Before proceeding with a decision, individuals should ask themselves if they would make the same commitment again, knowing what they now know. This reflection helps determine whether the commitment is rational or driven by external pressures.
\end{itemize}

\subsection*{Practical Applications}
\begin{itemize}
    \item \textbf{Marketing and Sales:} Companies can leverage the foot-in-the-door and low-balling techniques to increase sales, by first securing small commitments from customers and then encouraging larger purchases.
    \item \textbf{Goal Setting:} Individuals can use the principle of public commitment to enhance accountability and motivation for personal goals, such as fitness or academic achievements.
    \item \textbf{Leadership and Team Building:} Leaders can use the principle of effort justification to build stronger organizational commitment, encouraging employees to invest in meaningful tasks and thereby increasing their dedication to the company.
\end{itemize}
